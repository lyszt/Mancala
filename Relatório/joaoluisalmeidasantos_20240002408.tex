\documentclass[a4paper]{article}
\usepackage{inter}
\usepackage{parskip}
\usepackage{graphicx} 
\graphicspath{{Imagens/}{Relatório/Imagens/}}
\usepackage{setspace}
\usepackage{lettrine}
\usepackage{fancyhdr}
\usepackage[utf8]{inputenc}
\usepackage[portuguese]{babel}
\renewcommand{\baselinestretch}{1.0}
\setlength{\parskip}{0pt} 
\setlength{\parindent}{1em}
\fancyhf{} 
\renewcommand{\headrulewidth}{0pt} 
\rfoot{\thepage}
\usepackage{xcolor}
\definecolor{LightGray}{gray}{0.9}
\pagestyle{fancy}
\renewcommand{\thesection}{\Roman{section}.}
\renewcommand{\thesubsection}{}

\color{darkgray}
\usepackage{minted}

\usemintedstyle{tango}
\setminted{
    bgcolor=white,
    fontsize=\small,
    linenos=true,
    breaklines=true,
    frame=single,
    framesep=2mm,
    numbersep=5pt,
    tabsize=4
}
\usemintedstyle{pastie}


\begin{document}

\begin{titlepage}
    \textbf{Trabalho de Organização de Computadores 
    \\ Integrante:}\\
    João Luís Almeida Santos -- 20240002408
    \vfill
\end{titlepage}

\section{Introdução}

Este relatório descreve o desenvolvimento de uma implementação do jogo Mancala em Assembly RISC-V, executada no simulador RARS, como parte das atividades da disciplina de Organização de Computadores. A proposta do trabalho foi de simular o jogo de tabuleiro Mancala em formato de terminal.

\begin{figure}[h]
	\centering
	\includegraphics[width=1\textwidth]{Imagens/tabuleirojoga4setas.png}
	\caption{Representação do tabuleiro do Mancala}
	\label{fig:inicializacao}
\end{figure}

Este relatório descreve o desenvolvimento de uma implementação do jogo Mancala em Assembly RISC-V, executada no simulador RARS, como parte das atividades da disciplina de Organização de Computadores. A proposta do trabalho foi de simular o jogo de tabuleiro Mancala em formato de terminal.

\subsection{Inicialização e Estrutura de Dados}

Os primeiros passos do programa são dados na seção \texttt{.data}. Lá, são declaradas as variáveis, textos necessários para as funções de print, a vitória do jogador, o turno atual, etc. Além disso, todas as cavidades são inicializadas com o valor de 0, e a variável \textbf{SEED\_INIT} é criada com o valor 4. Esta variável pode ser alterada para mudar a funcionalidade do jogo.  
Cada mensagem para o usuário/jogador foi colocada em asciz. Com essas linhas especificamente foi possível criar o formato formatado do tabuleiro.  
Vale mencionar que em vários pontos, eu coloquei esses textos dentro de rótulos no .text, onde poderiam ser acessados pela função \textbf{print} para printar valores como se fosse em um for loop. Isso me permitiu diminuir o tamanho do arquivo no geral. O código no total deu 847 linhas, contando os comentários.

\subsection{Função de Inicialização do Tabuleiro}

A primeira função a ser chamada dentro do main é a de inicialização de tabuleiro. Ela é essencial para colocar os valores necessários dentro das cavidades para que o jogo efetivamente se inicie.  
Dentro dessa função, recebemos o número desejado em a0. Isso acontece apesar da existência do SEED\_INIT. Significa que a função não lê diretamente o SEED\_INIT. Ela recebe-o no início. Acredito que isso seja mais eficaz para caso queiramos mudar a lógica do tabuleiro de alguma forma.

\subsection{Lógica do Loop e Armazenamento}

De todo modo, a função prossegue. Ao receber o valor em a0, ela salva o valor em s0 para não perder em futuras chamadas de funções. Logo após, em \textbf{li s2, 5}, decidimos onde o loop vai parar enquanto estiver enchendo as cavidades. Não podemos chegar em 6 pois aí se localiza a cavidade de um dos jogadores.  
Iniciamos o Loop. Enquanto i não for igual a 5, continuamos. Chamamos a função auxiliar \textbf{armazena\_cavidade} para armazenar o valor no endereço i. A função de armazenar cavidade foi útil para evitar ter que ficar repetindo endereço inicial + i * 4 para acessar endereços toda hora. Todo esse processo é demonstrado pela figura \ref{fig:inicializacao}.

\subsection{Uso da Memória}

A memória foi dividida em duas seções principais. A \textbf{seção \texttt{.data}} contém variáveis globais, textos e vetores necessários para o jogo. A \textbf{seção \texttt{.text}} contém as funções principais e auxiliares. O código começa em \texttt{main}, inicializa o tabuleiro e executa o loop principal do jogo até o fim da partida.

\subsection{Uso dos Registradores}

Os registradores temporários (\texttt{t0–t6}) são usados para cálculos e comparações momentâneas. Os registradores salvos (\texttt{s0–s2}) armazenam valores persistentes entre chamadas de funções (ex.: valor inicial de sementes, contadores de loop). Os registradores de argumentos (\texttt{a0–a7}) seguem convenção padrão de chamadas. O registrador \texttt{sp} (stack pointer) é usado com as macros \texttt{startF} e \texttt{endF}, que salvam e restauram \texttt{ra}, \texttt{s0–s2}, 
e foram essenciais para o uso de funções.


\subsection{Funções Implementadas}
\begin{itemize}
    \item \texttt{main} — controla o fluxo principal do jogo, alternando entre os jogadores.
    \item \texttt{inicializar\_tabuleiro} — preenche as cavidades com o valor de \texttt{SEED\_INIT}.
    \item \texttt{mostra\_tabuleiro} — imprime o estado atual do tabuleiro.
    \item \texttt{player\_one\_turn} / \texttt{player\_two\_turn} — controlam as jogadas, incluindo roubo e turno extra.
    \item \texttt{distribute\_pellets} — distribui as sementes a partir da cavidade escolhida.
    \item \texttt{valida\_escolha} — garante que a cavidade selecionada é válida.
    \item \texttt{carrega\_cavidade}, \texttt{armazena\_cavidade}, \texttt{adiciona\_na\_cavidade} — manipulam valores no vetor \texttt{cavidades}.
    \item \texttt{soma\_valores\_j1/j2} — somam o total de sementes de cada lado.
    \item \texttt{incrementa\_vitoria\_j1/j2} — atualizam o placar.
    \item \texttt{print}, \texttt{print\_one\_string}, \texttt{print\_integer} — funções genéricas de saída via \texttt{ecall}.
    \item \texttt{read\_integer} — leitura numérica do usuário.
    \item \texttt{verifica\_vencedor} / \texttt{compara\_valores} — determinam o resultado final da partida.
\end{itemize}

\end{document}