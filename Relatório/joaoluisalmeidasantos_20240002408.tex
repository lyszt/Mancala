\documentclass[a4paper]{article}
\usepackage{inter}
\usepackage{parskip}
\usepackage{graphicx} 
\usepackage{setspace}
\usepackage{lettrine}
\usepackage{fancyhdr}
\usepackage[utf8]{inputenc}
\usepackage[portuguese]{babel}
\renewcommand{\baselinestretch}{1.5} 
\fancyhf{} 
\renewcommand{\headrulewidth}{0pt} 
\rfoot{\thepage}
\usepackage{xcolor}
\definecolor{LightGray}{gray}{0.9}
\pagestyle{fancy}
\renewcommand{\thesection}{\Roman{section}.}
\renewcommand{\thesubsection}{}

\color{darkgray}
\usepackage{minted}
\usemintedstyle{pastie}
\begin{document}

% sumário
\tableofcontents
\newpage


\section{Introdução}

Este relatório descreve o desenvolvimento de uma implementação do jogo Mancala em Assembly RISC-V, executada no simulador RARS, como parte das atividades da disciplina de Organização de Computadores. A proposta do trabalho foi de simular o jogo de tabuleiro Mancala em formato de terminal.

\subsection{Demanda do Trabalho}
O enunciado disponibilizado exige uma versão do Mancala com doze cavidades e dois poços, quatro sementes por casa no estado inicial, suporte a turnos extras, captura de sementes e detecção do fim de jogo. Dentro do código, procurei modularizar e abstrair o máximo da lógica, dado o uso de funções de Macro e a criação de funções que vaziam processos simples, como printar, printar em loop, ler inteiro, etc. 

\section{Explicação do Código}

\begin{figure}[h]
	\centering
	\includegraphics[width=0.85\textwidth]{Imagens/tabuleiro.png}
	\caption{Representação do tabuleiro do Mancala (CrazyGames)}
	\label{fig:tabuleiro}
\end{figure}



\section{Conclusão}


\end{document}
