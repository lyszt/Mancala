\documentclass[a4paper]{article}
\usepackage{inter}
\usepackage{parskip}
\usepackage{graphicx} 
\graphicspath{{Imagens/}{Relatório/Imagens/}}
\usepackage{setspace}
\usepackage{lettrine}
\usepackage{fancyhdr}
\usepackage[utf8]{inputenc}
\usepackage[portuguese]{babel}
\renewcommand{\baselinestretch}{1.5} 
\fancyhf{} 
\renewcommand{\headrulewidth}{0pt} 
\rfoot{\thepage}
\usepackage{xcolor}
\definecolor{LightGray}{gray}{0.9}
\pagestyle{fancy}
\renewcommand{\thesection}{\Roman{section}.}
\renewcommand{\thesubsection}{}

\color{darkgray}
\usepackage{minted}

\usemintedstyle{tango}
\setminted{
    bgcolor=white,
    fontsize=\small,
    linenos=true,
    breaklines=true,
    frame=single,
    framesep=2mm,
    numbersep=5pt,
    tabsize=4
}
\usemintedstyle{pastie}
\begin{document}

\begin{titlepage}
    \centering
    \thispagestyle{empty}
    {\Large Universidade Federal da Fronteira Sul\par}
    {\large GEX1213 -- Organização de Computadores\par}
    \vspace{2cm}
    {\huge\bfseries Relatório Técnico\par}
    \vspace{0.5cm}
    {\LARGE Jogo Mancala em Assembly RISC-V\par}
    \vspace{2cm}
    \begin{flushleft}
    \textbf{Integrante:}\\
    João Luís Almeida Santos -- 20240002408
    \end{flushleft}
    \vfill
    {\large Chapecó -- SC\\\today}
\end{titlepage}

% sumário
\tableofcontents
\newpage


\section{Introdução}

Este relatório descreve o desenvolvimento de uma implementação do jogo Mancala em Assembly RISC-V, executada no simulador RARS, como parte das atividades da disciplina de Organização de Computadores. A proposta do trabalho foi de simular o jogo de tabuleiro Mancala em formato de terminal.

\begin{figure}[h]
	\centering
	\setlength{\fboxrule}{1pt} 
	\setlength{\fboxsep}{4pt}  
	\fbox{\includegraphics[width=0.85\textwidth]{Imagens/tabuleiro.png}}
	\caption{Representação do tabuleiro do Mancala (CrazyGames)}
	\label{fig:tabuleiro}
\end{figure}


\subsection{Demanda do Trabalho}
O enunciado disponibilizado exige uma versão do Mancala com doze cavidades e dois poços, quatro sementes por casa no estado inicial, suporte a turnos extras, captura de sementes e detecção do fim de jogo. Dentro do código, procurei modularizar e abstrair o máximo da lógica, dado o uso de funções de Macro e a criação de funções que vaziam processos simples, como printar, printar em loop, ler inteiro, etc. 

\section{Explicação do Código}

\subsection{Inicialização}
Os primeiros passos do programa são dados na seção .data. Lá, são declaradas as variáveis, textos necessários para as funções de print, a vitória do jogador, o turno atual, etc. Além disso, todas as cavidades são inicializadas com o valor de 0, e a variável \textbf{SEED\_INIT} é criada com o valor 4. Esta variável pode ser alteradoa para mudar a funcionalidade do jogo.
\begin{minted}{asm}
    SEED_INIT:
    .word      4

    vitorias_j1:
        .word 0 
    vitorias_j2: 
        .word 0 
    turno_atual:
        .word 0 
    cavidades:
        .word      0, 0, 0, 0, 0, 0, 0, 0, 0, 0, 0, 0, 0, 0
    SEED_INIT = 4;
\end{minted}

Como visto na linha abaixo, cada mensagem para o usuário/jogador foi colocada em asciz. Com essas linhas especificamente foi possível criar o formato formatado do tabuleiro.
\begin{minted}{asm}
    
# Informações dos jogadores
titulo_jogador_1:
    .asciz     "Jogador 1\n"
titulo_jogador_2:
    .asciz     "Jogador 2\n"
texto_jogador_1:
    .asciz     "Escolha a cavidade [0-5]\n"
texto_jogador_2:
    .asciz     "Escolha a cavidade [7-12]\n"

vitoria_jogador_1:
    .asciz "Jogador 1 venceu!\n"
vitoria_jogador_2:
    .asciz "Jogador 2 venceu!\n"
empate:
    .asciz "Empate!\n"

texto_quer_jogar:
    .asciz "Quer jogar novamente? 1 para sim, 0 para não... \n"


mensagem_valor_invalido:
    .asciz     "Por favor escolha um valor válido!\n"
mensagem_roubo:
    .asciz     "Roubou as pedras do adversário!\n"
mensagem_turno_extra:
    .asciz     "Caiu na vala! Jogue de novo!\n"
mensagem_fim_jogo:
    .asciz     "Fim de jogo! Um lado ficou vazio.\n"
# Textos do tabuleiro
titulo_acima_jogador_1:
    .asciz     "                          0 <-- JOGADOR 1   5                          \n"
titulo_acima_jogador_2:
    .asciz     "                         12 <-- JOGADOR 2   7                          \n"
linha_horizontal:
    .asciz     "+----+----+----+----+----+----+----+----+----+----+----+----+----+----+\n"
linha_horizontal_meio:
    .asciz     "----+----+----+----+----+----+----+----+----+----+---+"
quadrado_vazio:
    .asciz     "|       |"
quadrado_esquerda:
    .asciz     "|   "
quadrado_direita:
    .asciz     "   |"
quebra_linha:
    .asciz     "\n"

    .align     2


\end{minted}

Vale mencionar que em vários pontos, eu coloquei esses textos dentro de rótulos no .text, onde poderiam ser acessados pela função \text{print} para printar valores como se fosse em um for loop. Isso me permitiu diminuir o tamanho do

\section{Conclusão}


\end{document}
