\documentclass[a4paper]{article}
\usepackage{inter}
\usepackage{parskip}
\usepackage{graphicx} 
\graphicspath{{Imagens/}{Relatório/Imagens/}}
\usepackage{setspace}
\usepackage{lettrine}
\usepackage{fancyhdr}
\usepackage[utf8]{inputenc}
\usepackage[portuguese]{babel}
\renewcommand{\baselinestretch}{1.5} 
\fancyhf{} 
\renewcommand{\headrulewidth}{0pt} 
\rfoot{\thepage}
\usepackage{xcolor}
\definecolor{LightGray}{gray}{0.9}
\pagestyle{fancy}
\renewcommand{\thesection}{\Roman{section}.}
\renewcommand{\thesubsection}{}

\color{darkgray}
\usepackage{minted}

\usemintedstyle{tango}
\setminted{
    bgcolor=white,
    fontsize=\small,
    linenos=true,
    breaklines=true,
    frame=single,
    framesep=2mm,
    numbersep=5pt,
    tabsize=4
}
\usemintedstyle{pastie}


\begin{document}

\begin{titlepage}
    \textbf{Integrante:}\\
    João Luís Almeida Santos -- 20240002408
    \vfill
\end{titlepage}

\section{Introdução}

Este relatório descreve o desenvolvimento de uma implementação do jogo Mancala em Assembly RISC-V, executada no simulador RARS, como parte das atividades da disciplina de Organização de Computadores. A proposta do trabalho foi de simular o jogo de tabuleiro Mancala em formato de terminal.

\begin{figure}[h]
	\centering
	\includegraphics[width=1\textwidth]{Imagens/tabuleirojoga4setas.png}
	\caption{Representação do tabuleiro do Mancala}
	\label{fig:inicializacao}
\end{figure}


\subsection{Demanda do Trabalho}
O enunciado disponibilizado exige uma versão do Mancala com doze cavidades e dois poços, quatro sementes por casa no estado inicial, suporte a turnos extras, captura de sementes e detecção do fim de jogo. Dentro do código, procurei modularizar e abstrair o máximo da lógica, dado o uso de funções de Macro e a criação de funções que vaziam processos simples, como printar, printar em loop, ler inteiro, etc. 

\section{Explicação do Código}

O programa inicia na seção .data, onde são declaradas as variáveis, textos para exibição, estado do jogo e a constante SEED_INIT = 4, que define o valor inicial das cavidades e pode ser alterada para modificar o comportamento do jogo. As mensagens ao jogador foram definidas com asciz, permitindo formatar o tabuleiro. Em alguns casos, esses textos foram movidos para rótulos no .text, acessados pela função print, o que reduziu o tamanho total do código (847 linhas, incluindo comentários).

A primeira função chamada no main é a de inicialização do tabuleiro. Ela recebe o valor inicial em a0 (mesmo havendo SEED_INIT) para maior flexibilidade na lógica do jogo. O valor é salvo em s0, e o loop é configurado com li s2, 5, para preencher apenas as cavidades válidas. Dentro do loop, a função auxiliar armazena_cavidade grava os valores nas posições corretas, simplificando o acesso aos endereços de memória. Esse processo é ilustrado na figura \ref{fig:inicializacao}.


\end{document}
